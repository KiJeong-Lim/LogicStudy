\documentclass[12pt]{paper}

\usepackage{amsfonts}
\usepackage{amsmath}
\usepackage{amssymb}
\usepackage{amsthm}
\usepackage{color}
\usepackage{stmaryrd}
\usepackage[bottom]{footmisc}
\usepackage[a4paper, left=20mm, right=20mm, top=20mm, bottom=20mm]{geometry}
\usepackage{graphicx}
\usepackage{kotex}
\usepackage{proof}
\usepackage{listings}
\usepackage{setspace}
\usepackage{enumitem}
\usepackage{array,multirow}
\usepackage{hyperref}
\usepackage[table]{xcolor}

\newcommand{\gnum}
{ \mathfrak{g} }

\newenvironment{context}[1][]
{ \noindent \textbf{{#1}.} }
{ \hfill $ \dashv $ }

\newcommand{\coqstyle}
{ \lstset
  { basicstyle = \footnotesize\ttfamily
  , breakatwhitespace = false
  , breaklines = true
  , frame = single
  , keywordstyle = \color{blue}
  , morekeywords =
    { as
    , Context
    , Defined
    , Definition
    , else
    , End
    , end
    , Example
    , Export
    , Fixpoint
    , forall
    , From
    , fun
    , if
    , in
    , Import
    , Inductive
    , Lemma
    , Let
    , let
    , Ltac
    , match
    , Proof
    , Prop
    , Qed
    , Record
    , Require
    , return
    , Section
    , Set
    , struct
    , then
    , Theorem
    , Type
    , with
    }
  , tabsize = 2
  }
}

\lstnewenvironment{coqcode}[1][]
{ \setstretch{1.0}
  \coqstyle
  \lstset{#1}
}
{ \setstretch{1.5} }

\title{힐베르트식의 일차논리 형식화}
\author{임기정}

\begin{document}

\maketitle

\section{구문론(syntax)}

\subsection{시그니처(signature)}

$\mathcal{F}, \mathcal{R}, \mathcal{C}$를
각각 함수 기호들의 집합, 관계 기호들의 집합, 상수 기호들의 집합이라고 하자.
$$n_\mathcal{F} : \mathcal{F} \to \mathbb{N} \quad \mathrm{and} \quad n_\mathcal{R} : \mathcal{R} \to \mathbb{N}$$
이 각각 함수 기호들의 arity와 관계 기호들의 arity에 대한 함수일 때, 순서쌍
$$ \Sigma := \left\langle \mathcal{F}, \mathcal{R}, \mathcal{C}, n_\mathcal{F} , n_\mathcal{R} \right\rangle $$
를 \underline{시그니처}라고 한다.
Coq으로는 다음과 같이 형식화할 수 있다.
\begin{coqcode}
Section FOL_SYNTAX.

Let arity := nat.
    
#[projections(primitive)]
Record folSignature : Type :=
  { function_symbols : Set
  ; relation_symbols : Set
  ; constant_symbols : Set
  ; function_arities : function_symbols -> arity
  ; relation_arities : relation_symbols -> arity
  }.
\end{coqcode}

\newpage

\subsection[short]{논리항(term)과 논리식(formula)}

시그니처 $$\Sigma = \left\langle \mathcal{F}, \mathcal{R}, \mathcal{C}, n_\mathcal{F} , n_\mathcal{R} \right\rangle$$가 주어졌다고 하자.
논리항은 다음과 같이 정의된다.
\begin{enumerate}
\item [(1)] (개체변수) $x \in \mathbb{N}$이면 $x$는 논리항이다.
\item [(2)] (상수) $c \in \mathcal{C}$이면 $c$는 논리항이다.
\item [(3)] (함수 적용) $f \in \mathcal{F}$일 때 $n := n_\mathcal{F} \left( f \right)$라 하면, 논리항 $t_1, \cdots , t_n$에 대하여 $f \left( t_1 , \cdots , t_n \right)$ 또한 논리항이다.
\item [($\ast$)] 단, 규칙 (1), (2), (3)만을 유한 번만 적용하여 얻을 수 있어야 한다.
\end{enumerate}
한편, 논리식은 다음과 같이 정의된다.
\begin{enumerate}
\item [(1)] (원자논리식) $R \in \mathcal{R}$일 때 $n := n_\mathcal{R} \left( R \right)$라 하면, 논리항 $t_1, \cdots , t_n$에 대하여 $R \left( t_1 , \cdots , t_n \right)$는 논리식이다.
\item [(2)] (등식) 논리항 $t_1$, $t_2$에 대하여 $t_1 \doteq t_2$는 논리식이다. 
\item [(3)] (부정) 논리식 $\varphi_1$에 대하여 $\dot\lnot \varphi_1$도 논리식이다.
\item [(4)] (함축) 논리식 $\varphi_1$, $\varphi_2$에 대하여 $\varphi_1 \dot\to \varphi_2$ 또한 논리식이다.
\item [(5)] (보편양화) 개체변수 $y \in \mathbb{N}$와 논리식 $\varphi_1$에 대하여 $\dot\forall y \varphi_1$도 논리식이다.
\item [($\ast$)] 단, 규칙 (1), (2), (3), (4), (5)만을 유한 번만 적용하여 얻을 수 있어야 한다.
\end{enumerate}
Coq으로는 다음과 같이 형식화할 수 있다.
\begin{coqcode}
Definition ivar : Set := nat.

Context {Sigma : folSignature}.

Inductive trm : Set :=
  | IVar_trm (x : ivar) : trm
  | Cnst_trm (c : Sigma.(constant_symbols)) : trm
  | Fapp_trm (f : Sigma.(function_symbols)) (ts : trms (L.(function_arities) f)) : trm
with trms : arity -> Set :=
  | O_trms : trms 0
  | S_trms (n : arity) (t : trm) (ts : trms n) : trms (S n).

Inductive frm : Set :=
  | Atm_frm (R : Sigma.(relation_symbols)) (ts : trms (L.(relation_arities) R)) : frm
  | Eqn_frm (t1 : trm) (t2 : trm) : frm
  | Neg_frm (p1 : frm) : frm
  | Imp_frm (p1 : frm) (p2 : frm) : frm
  | All_frm (y : ivar) (p1 : frm) : frm.
\end{coqcode}

\end{document}
